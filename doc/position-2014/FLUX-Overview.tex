It is a serious undertaking 
to design the production version of a new 
RM framework that can realize our proposed conceptual models.
Thus, \flux\ team has charted its course through a series of steps
meant to inform our design process. 
We began with a functional design that includes an
internal review with various stakeholders including 
developers at the operating systems level
as well as run-time and applications level.
Then, our initial research thrust was the run-time system of \flux.

%Then, this notion provided us with a concise 
%mechanism by which we can relate groups of processes
%to underlying resources (e.g., containing certain processes 
%to a set of resesources) as well as relate groups
%one another (e.g., synchronizing tool processes to 
%MPI processes). 
%
For our run-time, we have developed prototypes of our communication framework 
called Communication Message Broker (CMB) 
and a scalable key value store (KVS) service.
They comprise essential scalable building blocks of our run-time system, 
and we have begun to build various run-time services and also 
to port user-level run-time software programs to them.
Further, we introduced the notion of lightweight 
job (LWJ) as a new way to organize distributed or parallel processes.
Using this notion, a wide range of run-time software
programs such as parallel programming models,
tools and middleware can all leverage our run-time services 
to build on the strength of one another 

\ifcomments
\marginpar{\tiny BS: Do you want to now make a case for why KVS is going
to be an important building block?  Or just launch into the next
two sections on CMB and V'S?}
\marginpar{\tiny DA: I think I made a case. Let me know if not}
\fi
