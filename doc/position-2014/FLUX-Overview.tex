To explore the feasibility of the \flux design, we implemented prototypes
of the two key components in the \flux run-time environment. 
%
%our conceptual design, we started to 
%explore the core run-time components of \flux\
%through rapid prototyping.
%%RJMS framework that can realize the new paradigm.
%%Thus, \flux\ team has engaged in a series of steps
%%meant to inform our design process. 
%%We began with a functional design, reviewed
%%by a small group of stakeholders including 
%%systems, run-time, and applications developers,
%%as well as data center management.
%%Subsequently, we initiated a prototyping phase for the \flux\ run-time.
%%Then, this notion provided us with a concise 
%%mechanism by which we can relate groups of processes
%%to underlying resources (e.g., containing certain processes 
%%to a set of resesources) as well as relate groups
%%one another (e.g., synchronizing tool processes to 
%%MPI processes). 
%%
%These components include 
%the communication framework 
%called the Comms Message Broker (CMB) 
%and a scalable key value store (KVS) service.
%Together, they form %They are an essential 
%the backbone that serves 
%independent RJMS services per each job according to our
%unified job model and also embodies our
%common scalable communication infrastructure model, the most critical issues with respect to scalability.

%Further, we developed support for our concept of the lightweight 
%job (LWJ) as a new way to organize distributed or parallel processes.
%Using this notion, a wide range of run-time software
%programs such as parallel programming models,
%tools, and middleware can leverage our common run-time services 
%to build upon \flux\ and each other.

%\ifcomments
%\marginpar{\tiny BS: Do you want to now make a case for why KVS is going
%to be an important building block?  Or just launch into the next
%two sections on CMB and V'S?}
%\marginpar{\tiny DA: I think I made a case. Let me know if not}
%\fi
