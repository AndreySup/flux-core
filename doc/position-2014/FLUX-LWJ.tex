\subsection{Lightweight Job (LWJ)}
CMB and KVS are the scalable building blocks not only for our run-time services 
but also other key run-time software such as parallel programming models, 
tools, and middleware. These programs commonly employ distributed processes and need a 
flexible and concise mechanism to relate their processes to underlying resources 
(e.g., containing certain processes to a set of resources) as well as relate 
them to other processes. 

The traditional approach models those processes as
a set of compute steps---e.g., job steps. However, this model 
is MPI-centric and too static to support emerging types
of run-time patterns arising from dynamic workload, tools, and middleware.
Thus, we introduce a more flexible concept called the lightweight job (LWJ).
An LWJ is a group
of processes with a distinct function that has its own resource confinement. 
For example, all of the parallel
processes of an MPI application may form a single compute LWJ; all of the distributed processes
of a parallel debugger program may form a tool LWJ that should be logically separate from the
compute LWJ; further, the compute LWJ may dynamically refine itself into several 
sub groups to serve independent power-capping functions to different subsets of its processes.
We currently use the KVS to organize LWJ information hierachically.
