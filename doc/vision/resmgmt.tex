\section{Resource Management}

FIXME: pulled in an email, fix style

I propose that the core component of the "resource management"
part of NGRM be a persistent, global database that I've been
calling the "Resource Inventory". The Resource Inventory contains
a superset of resources managed by the system, at at the highest
level acts as the configuration repository for the NGRM system.
The resource inventory allows for arbitrary tags on resources,
and is not associated with any "instance" of the resource
manager.

I propose the following initial features of the resource inventory:

\begin{itemize}
\item{Arbitrary "tagging" of resource objects. Tags can be used later
   to define searches for objects. E.g. give me all the resources
   with the tag 'compute' and 'node' and 'idle', etc.}

\item{A subscribe interface with filtering so apps using the DB can
   subscribe to interesting changes and updates in the DB.}

\item{An API upon which sysadmin and other tools could be developed.}
\end{itemize}

Resource manager instances/jobs do not interact directly with
the Resource Inventory. I propose that each RM instance has associated
with it a rw "cache" of its subset of the resource inventory DB,
which we will call simply the resource DB. The resource DB
'listens' for changes of interest from its parent DB using
the subscribe interface, and additionally publishes its own
changes such that a parent/child or other application can
be notified of data changes to the cache (e.g. node is allocated
to sub-job, node is marked down, user pushes some tags of
interest into the cache).

At the end of a job, a configurable set of data in the
resource DB would be pushed back up to the parent. This
allows accumulated data and job/user specific tags from
each job to percolate back up the system, possibly all
the way to the persistent Resource Inventory.

\newpage
\subsection{Resource Management WBS}

\begin{longtable}{|p{1cm}|p{10.2cm}|p{1cm}|p{1cm}|p{1.8cm}|}\hline
  \textbf{Item} & \textbf{Description}
                & \textbf{Deliv}\footnote{SD = software drop,
                        DR = design review, V = viewgraphs, D = document}
                & \textbf{Weeks} & \textbf{Dep} \\
  \hline
  2.1.  & "Functional", high-level model for how the system above
          would work in \ngrm, including models for Resource Inventory,
          Job Data, Queues, Scheduling.
        & V
        & 
        & \\
  \hline
  2.2.  & Resource and Job DB APIs (design and prototype).
        & DR, V
        & 
        & 2.1\\
  \hline
  2.3.  & Resource Inventory (design and prototype)
        & DR, V
        & 
        & 2.1\\
  \hline
  2.4.  & Job Database (design and prototype)
        & DR, V
        & 
        & 2.1\\
  \hline
\end{longtable}
