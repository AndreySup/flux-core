\section{NGRM Job Scheduler}

This section presents the requirements and design for the job
scheduling component of the \ngrm.

The \ngjs\ is responsible for scheduling computing resources to users'
jobs.  Users submit to the scheduler requests for resources to run
their job.  The scheduler implements management's policy to decide
when and where to allocate the resources for each job.

This section presents the requirements for the \ngjs, a rough design
which meets those requirements, and a work breakdown structure for
developing the scheduler component.

\subsection{Motivation}

Scheduling batch jobs across a collection of networked computing
resources started in the 1990's with Livermore Computing's DPCS (later
known as LCRM).  It received users' job requests, selected a cluster
for each job, then dispatched the job to that cluster's resource
manager.  The Moab Workload Manager which replaced LCRM essentially
provided the same functionality.  And while SLURM provides some grid
functionality, it never matured enough to allow it to replace Moab for
production use.

The \ngjs\ will provide new functionality not available in any
commercial or open source project.  The \ngjs\ will schedule jobs
across resources in a computing center without regard to traditional
cluster boundaries.  A job will be able to request resources
containing a common feature (like connectivity to the same high speed
switch) or fitting within a limited power envelope.

In addition, the \ngjs\ will be designed to be recursive.  It will
have the capability to launch another instance of itself within the
context of a job.  The \ngjs\ will support job priority plugin modules
allowing each recursive instance of the scheduler the ability to run a
different priority plugin.  In so doing, the \ngjs's scheduling
capabilities will range from scheduling all resources in the center to
scheduling jobs on dedicated resources (DATs) to scheduling what were
formerly known as job steps.

Most importantly, the traditional boundaries between a job scheduler
and the resource manager will be redefined under \ngrm.  Instead of a
resource manager that manages every resource of a cluster, the
\ngrmfull\ will be instantiated on the fly by the \ngjs\ and manage
only the resources the scheduler allocates to the job (or recursive
job).

In order to continue to meet the needs of LC users, the \ngjs\ must
continue to provide all the services that Moab currently provides,
but...
\begin{itemize}
  \item More quickly and efficiently
  \item More accurately
  \item More reliably
  \item More easily
  \item More intuitively
  \item More flexibly
  \item More securely
  \item More transparently
  \item And require minimal intervention and oversight
\end{itemize}

\subsection{Requirements}

While a more detailed list of requirements is presented here
<ToBeDone>, the following provides an overview of the functionality
that the \ngjs\ will be expected to deliver.

\subsubsection{Fundamental Requirements}

The following is the most definitive list of scheduling requirements.
A scheduler is not a scheduler unless it can do these:

\begin{itemize}
  \item Maintain current resource inventory including state, health
    and availability of all resources to be scheduled
  \item Receive job allocation requests
  \item Prioritize each job
  \item Schedule each job based on multiple resource requirements
  \item Backfill lower priority jobs whenever possible
  \item Allow for dynamic job growth and reduction
  \item Preempt running jobs to free up resources needed by higher priority jobs
  \item Provide status of all jobs with estimates of when each job will run
  \item Allow for recursive scheduling
\end{itemize}

\subsubsection{Further Scheduler Requirements}

As part of its scheduling duties, the scheduler must also accommodate
user requests and provide versatile administration capabilities.

\begin{itemize}
  \item Respond to requests to hold, modify or cancel jobs
  \item Schedule jobs across the center
  \item Support complex job dependencies
  \item Restrict some operations based on roles
  \item Accept job reservations
  \item Originate and manage resource manager instances
  \item Maintain some job statistics (e.g., resource usage per user
    per account per cluster)
  \item Process live configuration updates
  \item Save complete state and recover fully from a restart
\end{itemize}

\subsubsection{Policy Enforcement}

The scheduler implements the center's policies for providing access to
its computing resources.  As part of this duty, the scheduler must

\begin{itemize}
  \item Reject job submissions for jobs which cannot or will never run
  \item Remove jobs that exceed time limits
  \item Honor limits imposed on users, groups, projects (banks), etc
  \item Honor service level agreements and service quality requests
\end{itemize}
