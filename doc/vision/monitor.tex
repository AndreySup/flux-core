\section{Monitoring}

\ngrm\ monitoring consists of three main components:
a plugin system which allows user and system defined code to execute
periodically to check for faults or take data samples,
a monitoring console which aggregates the state of resources within a
comms sesssion and presents it in the form of a web page for users and
support staff, and a database interface used to store data for
offline analysis.
These subsystems are layered upon the comms framework event messaging and
aggregation/reduction services.  The general approach is to allow monitoring
to be customized within a session, with the parent session delegating
the responsibility for monitoring to its children.

\subsection{Monitoring Plugin System}

The monitoring plugin system provides a facility for arbitrary code to be
periodically executed across a session.  To minimize disruption to
bulk-synchronous workloads, this execution is synchronized by the 
session scheduling trigger event.  The set of active plugins is
under the control of the session owner, with some reasonable
defaults provided that can be overridden.

The details of the plugin structure is a design activity, however,
an example of a possible solution is for plugins to take the form of
Lua scripts loaded into the CMB state database
(thus shared across the session) and executed in the context of
a monitoring daemon with an embedded Lua interpreter.

Plugins have three main functions: data source, data reduction,
and data sink.  The data source function is driven by the scheduling
trigger and may directly publish events, for example if a monitored
value exceeds a threshold, or produce structured data for the
aggregation/reduction network.
The data reduction function aggregates structured data from instances
of the plugin's source function.
For example, a plugin that monitors the total amount of memory used
by the session might register a reduction function that takes the sum of
arriving samples.
Finally, plugins can register a function to sink data at the monitoring
console, rendering it for display, or at points in the session that interface
with the monitoring database for persistent storage.

\subsection{Monitoring Console}

The monitoring console runs on the session control node and provides
a centralized point for monitoring each session.
An HTTP interface is provided which includes links to the control nodes
of child sessions, thus the entire hierarchy of sessions can be navigated
from the root session with a web browser.
The console displays information about the session that comes directly
from the comms CMB, including its time of inception, node membership,
node liveness, users, and scheduling trigger period.

The set of active plugins and the views created by them are represented.
Events generated by plugins are displayed in a scrollable region.

All the information that is available via the monitoring console web page
is also available via command line tools to a shell user on the session
control node.

\subsection{Monitoring Database Interface}

The monitoring database will be a large, centralized, persistent database
that stores structured log data for analysis and data provenance.
Log data is generated by monitoring plugins, but may come from other sources
such as syslog, the \ngrm\ runtime, resource manager, or scheduler.

One reason this database is centralized is to allow queries
to be constructed that correlate failures or performance anomalies
across domains; for example, jobs running slow or producing incorrect
results because of file system problems.  The database will directly
benefit system administrators and suport staff who currently perform
these tasks manually.

By preserving contextual information surrounding the execution of a job,
we enable questions to be asked later on when the same job produces
different results.  This is a central goal of provenance.

The database will need some schema design in order to enable queries
that are operationally useful.  For example, RAS metrics may be obtained
on hardware component failures if we are careful to log enough information
to identify the component (e.g. a double bit memory error at address \#xxxx
versus a failure of dimm with model and serial number).

This effort will leverage the work of an in-progress LDRD
feasbility study\cite{LogLDRD}.

\subsection{Monitoring WBS}

\begin{longtable}{|p{1cm}|p{10.2cm}|p{1cm}|p{1cm}|p{1.8cm}|}\hline
  \textbf{Item} & \textbf{Description}
                & \textbf{Deliv}\footnote{SD = software drop,
                        DR = design review, V = viewgraphs, D = document}
                & \textbf{Weeks} & \textbf{Depend} \\
  \hline
  \hline
  \multicolumn{5}{|l|}{3.3. \textbf{Monitoring Plugin System}} \\
  \hline
  3.3.1.& Design/prototype plugin system including structured log format
	  and event naming.  (See also possible CIFTS/FTB tie-in in runtime).
          (See also Meyer monitoring project).
	& DR
	&
	& comms \\
  \hline
  3.3.2.& Implement plugin system.
        & SD
        &
        & 3.3.1 \\
  \hline
  3.3.3.& Document process for creating monitoring plugins.
        & D
        &
        & 3.3.1 \\
  \hline
  3.3.4.& Design/prototype a set of default plugins including plugins
          for instrumenting jobs to gather "implicit provenance" such as
          file accesses.
        & DR
        & 
        & 3.3.1 \\
  \hline
  3.3.5 & Implement set of default plugins.
        & SD
        &
        & 3.3.4 \\

  \hline
  \multicolumn{5}{|l|}{3.4. \textbf{Monitoring Console}} \\
  \hline

  3.4.1.& Design/prototype HTTP/REST monitoring console.
          (Long/Martinez Lorenz team)
          (See also Meyer monitoring project).
	& DR
	&
	& 3.3.1 \\
  \hline
  3.4.2.& Implement monitoring console.
	& SD
	&
	& 3.3.1, 3.4.1 \\
  \hline
  3.4.3.& Design/prototype Lorenz integration.
	  Think about how monitoring console integrates with the myllnl
	  dashboard experience.
          (Long/Martinez Lorenz team)
	& DR
	&
	& 3.4.1 \\
  \hline
  3.4.4.& Design/prototype skummee integration.
          How will \ngrm\ integrate with ops monitoring view?
          How will out of band monitoring (IPMI, DDN's, etc) integrate with
	  monitoring console?
          (Meyer monitoring project).
	& DR
	&
	& 3.4.1 \\
  \hline
  3.4.5.& Implement Lorenz/skummee integration.
	& SD
	&
	& 3.4.3, 3.4.4 \\
  \hline
  \multicolumn{5}{|l|}{3.5. \textbf{Monitoring Database Interface}} \\
  \hline
  3.5.1.& Study available NoSQL databases for 100K node scalability
          and appropriate query interface.
          Use offline log data to investigate system diagnostic capability
          and prototype queries.
          (Gamblin/Mohror HPC Data Analytics FY12 LDRD)
        & V
        & 
        & LDRD \\
  \hline
  3.5.2.& Implement prototype database tied to live log sources.
          Study scalability and develop queries.
          (Gamblin/Mohror HPC Data Analytics FY12 LDRD)
          (See also: Faaland SPLUNK deployment).
        & DR
        & 
        & 3.5.1 \\
  \hline
  3.5.3.& Design/prototype access-role based security.
        & DR
        & 
        & 3.5.2 \\
  \hline
  3.5.4.& Design/prototype schemas and queries for reporting
          RAS metrics of interest to center management.
        & DR
        & 
        & 3.5.2 \\
  \hline
  3.5.5.& Design/prototype procedure for sanitizing and releasing data
	  for research study and citation.
        & DR
        & 
        & 3.5.2 \\
  \hline
  3.5.6.& Design/prototype schema for job logs and queries for
          associating job data, system log data, etc..
        & DR
        & 
        & RM, 3.5.2 \\
  \hline
  3.5.7.& Implement database.
        & SD
        & 
        & 3.5.2, 3.5.3, 3.5.4, 3.5.5, 3.5.6 \\
  \hline
\end{longtable}


