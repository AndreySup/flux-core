\section{File System Provisioning and System Monitoring}

This section describes the system provisioning and system monitoring
subsystems of \ngrm.  By coupling these systems to the resource manager,
we achieve our goals of mitigating system noise, improving job launch time,
and promoting reproduceability and data provenance.

\subsection{File System Provisioning}

We use the term "file system provisioning" to describe how executables
and other read-only data are provided to compute nodes.
A provisioning system based on immutable datasets is proposed.
Immutable datasets are collections of files that can be aggressively cached
without revalidation,
and accurately referenced in a job record for data provenance.
In the proposed system, a dataset is stored in a container,
actually itself a file formatted as a local file system such as ext4 or
squashfs.  The container, which is stored in a special dataset repository,
is exported as a read-only network block device,
then mounted as a read-only, local file system on compute nodes,
where the files that comprise the dataset are individually accessible.
The system can be used to provision the root file system,
or any other file system provided its content can be stored immutably
in a dataset container and managed in the dataset repository.

The provisioning system consists of the
distributed network block device,
dataset repository,
tools for creating and managing datasets,
and the resource manager interfaces for assigning datasets to jobs
such that they become part of a job's runtime environment and
are referenced in the job's historical record.


\subsubsection{Distributed Network Block Device}

A network block device driver based on 9P protocol called 9nbd\cite{9nbd}
was prototyped in August 2012.
9nbd, which at a high level has functionality similar to iSCSI or SRP,
leverages the existing kernel 9P transport to access a backing file
in a remote 9P file system.
A hierarcically distributed, shared network block device can be built
using 9nbd and any number of levels of chained 9P diod\cite{diod}
I/O forwarding servers, fanned out in a hierarchy as depicted in
Figure~\ref{Fig9nbd}.

\paragraph{Performance Outlook}
A diod I/O forwarding server will read the working set of the container
file in once and thereafter serve it from its page cache to multiple clients.
At each compute node,
the block device driver naturally caches its working set in the buffer cache,
which is optimized for many common file system usage patterns,
such as path search and dynamic library loading.
A study of parallel path search\cite{BlkDevPathSearch}
using ext4, the {\em loopback} block device, one level of diod I/O forwarding
with a fanout of 82 to 1, and a single NFS server,
showed flat scaling to 82 nodes, the largest scale attempted,
while NFS began to slow exponentially at eight nodes.
This result indicates that the distributed block device based on 9P has
great promise to improve job launch times over NFS, and the flexible
hierarchical architecture should easily scale to 100K nodes, although
testing with more than one level of diod forwarding may identify some
additional work required to get there.

\paragraph{System Noise Outlook}
The distributed block device requires no cache management as the
data is read-only throughout the system.  Unlike NFS (even mounted read-only),
compute nodes will not attempt to asynchronously revalidate cached
attributes or data, since the entire dataset is immutable.
Data in the buffer cache is never dirty, and can be thrown out and at
any point should the application require the memory, and reread on demand later.
Therefore, the approach is very friendly to applications running on compute
nodes, with a system noise footprint that can be near zero.

\paragraph{Security}
The 9nbd prototype and diod server implement MUNGE authentication
over the built-in TCP transport plugin, and no integrity or privacy.
The \ngrm\ comms framework security model should be leveraged,
by adding an \ngrm-specific kernel 9P transport plugin,
to obtain integrity and privacy for I/O forwarding within a comms session,

\paragraph{Resiliency}
The 9nbd prototype implements single server resiliency, where the
9nbd block device driver can re-establish the transport and replay
9P connection state to a new server instance should the server reboot.
The client blocks while this recovery takes place.
The \ngrm\ comms framework CMB should be leveraged,
by adding an \ngrm-specific kernel 9P transport plugin,
to obtain multi-server resiliency based on storing possible diod server
identities in the CMB state and tracking liveness changes.
The client can thus recover more quickly by leveraging the CMB.

\paragraph{Root File System Special Needs}
Provisioning the system root file system to diskless nodes requires
some additional care and is somewhat distribution specific.
Root has to be mounted from a limited initramfs environment, which
implies that MUNGE and 9nbd have to be functional in that environment.
For distributions based on Red Hat Enterprise Linux, a {\em dracut}
module is needed for both MUNGE and 9nbd.  Once root is mounted, it can
remain read-only, or more conveniently, it can be combined with a {\em zram}
memory-based block device using {\em dm-snapshot}, making it appear
writable, with any changes going to RAM.  The dracut modules and the
dm-snapshot/zram support can be added to the nfsroot\cite{nfsroot}
software package.

\subsubsection{Dataset Repository}

Dataset containers can initially be stored in NFS, Lustre, or a local file
system and managed by system administrators using simple scripts.
However, there are a few practical problems with this.
One is that a large number of nearly identical containers may proliferate
and use too much space.  Another is that there is no mechanism to take a
reference on a dataset such that the reference is guaranteed to refer to
the same content forever.
Finally, the intent is to allow users to manage the content of their datasets,
but user access must be controlled such that containers are always
valid file system images and container metadata is properly filled in.
These three factors motivate the need for a specialized dataset repository.

\paragraph{Copy-on-write}
A large number of nearly identical containers may proliferate, for example
if a dataset is undergoing iterative changes during development.
Ideally the file system storing dataset containers will support
{\em thin copies} or block level {\em de-duplication} so that only one
copy of unchanged content is stored.

\paragraph{Content Addressable}
To prevent datasets from changing while mounted, which would trigger
kernel panics, and from run to run, which would affect provenance and
reproduceability, the file system storing dataset containers should
be content addressable.  For example, instead of a file name, a cryptographic
hash of its content is used to refer to a container file.
The file system or the repository interface should enforce this atomically.

\paragraph{Container Format and Repository Model}
Containers must contain valid file system images in order to avoid
mount errors or kernel panics, and certain metadata should be associated
with containers to record information about how it was created.
The resource manager will take references on containers when they are used
and when they are referenced by a job's provenance record.
Thus the dataset repository is more than a file system that has
content addressability and copy-on-write.
It must enforce provsioning system policy over container additions and
removals and ensure that it contains only valid containers.

\subsubsection{Tools for Creating and Managing Datasets}

Tools similar to those provided for package management are required to
create valid containers with proper metadata from collections of files,
for managing and updating containers, and for interacting with the repository.
Stakeholders such as system administrators and code teams who might
be interested in creating datasets should be consulted to understand
their workflow and tool needs.  

\subsubsection{Resource Manager Interface}

The provisioning system is intended to be used in conjunction with
the \ngrm\ runtime's capability of configuring the file system
namespace (e.g. set of mount points) on a custom basis for each job. 

FIXME: launching a custom I/O forwarding topology within the job.

FIXME: taking a reference on a container while it is mounted and
from job log for provenance.

\subsection{Monitoring}

\newpage
\subsection{Monitoring/Provenance WBS}

\begin{longtable}{|p{1cm}|p{10.2cm}|p{1cm}|p{1cm}|p{1.8cm}|}\hline
  \textbf{Item} & \textbf{Description}
                & \textbf{Deliv}\footnote{SD = software drop,
                        DR = design review, V = viewgraphs, D = document}
                & \textbf{Weeks} & \textbf{Depend} \\
  \hline
  \hline
  \multicolumn{5}{|l|}{3.1. \textbf{Distributed Network Block Device}} \\
  \hline
  3.1.1.& Complete initial 9nbd network block device development, including
          backport upstream 9P transport fixes, refine/test 9nbd resiliency
          code, submit for LKML integration.
        & SD
        & 
        & \\
  \hline
  3.1.2.& (root file system only) Add MUNGE key bootstrap and 9nbd root
	  support to nfsroot dracut module.
        & SD
        & 
        & \\
  \hline
  3.1.3.& (root file system only) Backport zram TRIM support so space
	  can be reclaimed in zram-based root file systems.
        & SD
        & 
        & \\
  \hline
  3.1.4.& Design/develop initial container format and rudimentary scripts
	  for creating containers for manually managed NFS dataset repository.
        & SD
        & 
        & \\
  \hline
  3.1.5.& Production release of distributed network block device
	  with root file system support.
	  Support booting up to 4000 nodes from a container stored
	  on a Netapp NFS filer, and forwarded via one level of diod
	  I/O forwarding.
	  Implement single node resiliency and MUNGE authentication
	  (no privacy or integrity).
        & SD
        & 
        & 3.1.1, 3.1.2, 3.1.3, 3.1.4\\
  \hline
  3.1.6.& Performance study at scale of direct NFS vs distributed block
	    device, including Pynamic, pathwalk, and FTQ.
        & V
        & 
        & 3.1.5 \\
  \hline
  3.1.7.& Explore best way to remount multiple levels of diod servers.
          Use v9fs with cache=loose?  Is more work needed to allow
	  arbitrary levels of I/O forwarding?
        & V
        & 
        & \\
  \hline
  3.1.8.& Design/prototype 9P transport plugin that leverages the \ngrm\ 
	  comms framework for resiliency and privacy/integrity.
        & DR
        & 
        & \\
  \hline
  3.1.9.& Implement 9P transport plugin.
        & SD
        & 
        & 3.1.8\\
  \hline
  \multicolumn{5}{|l|}{3.2. \textbf{Dataset Repository, Tools, and \ngrm\ Integration}} \\
  \hline
  3.2.1.& Design/prototype SLURM SPANK plugin that allows users
          to select among available root images.  Manage private file system
          namespace for user, perform post-mount configuration, perform uid
          mapping.  Develop strategy for handling mounts such as home
          directories that would be overmounted by a new root.
        & DR
        & 
        & 3.1.5 \\
  \hline
  3.2.2.& Design/prototype container format and tools for
	  creating/managing datasets.
	  Consult with stakeholders to understand workflow.
	  Demo prototype and iterate.
        & DR
        & 
        & \\
  \hline
  3.2.3.& Design/prototype dataset repository with copy-on-write/dedup,
	  content addressability, repo contract enforcement.
        & DR
        & 
        & \\
  \hline
  3.2.4.& Implement tools for creating/managing datasets.
        & SD
        & 
        & 3.2.2 \\
  \hline
  3.2.5.& Implement dataset repository.
        & SD
        & 
        & 3.2.3 \\
  \hline
  3.2.6.& Design/prototype \ngrm\ integration including
	  private file system namespace management,
          RM management of dataset references during execution and in job
	  record,
	  RM launch of private diod daemons.
	  (co-design with \ngrm\ runtime?)
        & DR
        & 
        & 3.1.5, 3.2.1, 3.2.3, 3.2.4 \\
  \hline
  3.2.7.& Implement \ngrm\ integration.
        & SD
        & 
        & 3.2.6 \\
  \hline
  \multicolumn{5}{|l|}{3.3. \textbf{Monitoring Database}} \\
  \hline
  3.3.1.& Study available NoSQL databases for 100K node scalability
          and appropriate query interface.
          Use offline log data to investigate system diagnostic capability
          and prototype queries.
          (Gamblin/Mohror HPC Data Analytics FY12 LDRD)
        & V
        & 
        & LDRD \\
  \hline
  3.3.2.& Implement prototype database tied to live log sources.
          Study scalability and develop queries.
          (Gamblin/Mohror HPC Data Analytics FY12 LDRD)
          (See also: Faaland SPLUNK deployment).
        & V
        & 
        & 3.3.1 \\
  \hline
  3.3.3.& Design/prototype access-role based security.
        & DR
        & 
        & 3.3.2 \\
  \hline
  3.3.4.& Design/prototype schemas and queries for reporting
          RAS metrics of interest to center management.
        & DR
        & 
        & 3.3.2 \\
  \hline
  3.3.5.& Design/prototype procedure for sanitizing and releasing data
	  for research study and citation.
        & DR
        & 
        & 3.3.2 \\
  \hline
  3.3.6.& Design/prototype schema for job logs and queries for
          associating job data, system log data, etc..
        & 
        & 
        & RM, 3.3.2 \\
  \hline
  3.3.7.& Implement production monitoring kit.
          Document procedures and practices for extending and using the system.
          (opportunity for multiple publications here) 
        & SD, D
        & 
        & 3.3.2 3.3.3, 3.3.4, 3.3.5, 3.3.6 \\
  \hline
  \multicolumn{5}{|l|}{3.4. \textbf{Monitoring Framework}} \\
  \hline
  3.4.1.& Design/prototype log aggregation and routing to log database
	  based on \ngrm\ comms.
        &
        & DR
        & comms \\
  \hline
  3.4.2.& Design/prototype event-based monitoring framework based on
	  \ngrm\ comms.  (Coordinate with Meier monitoring improvement project)
        &
        & DR
        & comms \\
  \hline
  3.4.3.& Design/prototype plugin interface that allows
          users to customize event based monitoring within a job.
        &
        & DR
        & 3.4.2 \\
  \hline
  3.4.4.& Design/prototype plugin interface that allows
          users to customize log generation and aggregation within a job.
        &
        & DR
        & 3.4.1 \\
  \hline
  3.4.5.& Design/prototype a plugin interface for instrumenting jobs
          to gather "implicit provenance" such as file accesses.
        & DR
        & 
        & 3.3.2 \\
  \hline
  3.4.6.& Implement monitoring framework.
          Document procedures for writing plugins and extending the system.
        &
        & SD, D
        & 3.4.1, 3.4.2, 3.4.3, 3.4.4, 3.4.5. \\
  \hline
\end{longtable}

