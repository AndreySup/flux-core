\section{Terminology}

\paragraph{comms session}
An established communication association among a set of nodes that
enables secure, scalable, elastic, and fault-resilient communication
services for an \ngrm\ {\bf instance}.
A comms session is identified by its private DNS name, e.g. s1.\ngrm.
or s1.s1.\ngrm.

\paragraph{confinement}
The enforcement of resource allocations such that an {\bf instance} cannot
consume more than was allocated.  Confinement of an instance to assigned
resources is the responsiblility of the parent and, in the case of inherited
confinment, grandparent instance(s).

\paragraph{control node}
A distinguished node within a {\bf comms session} which holds the master
{\em comms state}, is the root of the session's aggregation/reduction tree,
and which performs gateway functions with the parent session.

\paragraph{instance}
An independent set of resource manager services configured to manage
a set of resources.
Instances are created dynamically and recursively.
We refer to the bootstrap {\em root instance} which contains all resources,
and speak of {\em parent}, {\em child}, and {\em sibling} relations between
instances.
An instance is identified by its {\bf comms session} name,
e.g. s1.\ngrm.  or s1.s1.\ngrm.

\paragraph{job}
A time-bounded allocation of resources.  A job is submitted to a
particular {\bf instance}.  When it executes, a child instance is
created to contain the job.  As a special case the {\em root job} 
runs perpetually without a time bound.
An job is identified by its {\bf instance} name,
e.g. s1.\ngrm.  or s1.s1.\ngrm.

\paragraph{lightweight job} (LWJ) 
A {\bf job} submitted and recorded in the current {\bf instance}
which does not result in a new instance.
